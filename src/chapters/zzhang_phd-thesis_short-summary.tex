
\documentclass{book}
\begin{document}
% English version
This thesis aims to understand the genetic basis of complex human traits using systems biology (SB).
SB exploits quantitative measurements, computational models, and high-throughput screening to decipher biological systems.

Next-generation sequencing technologies allow profile human molecular traits cost-effectively at large scales, which brought biology research into the multi-omics era.

Omics studies initiated with genomics.
Combining genomics with transcriptomics and epigenomics enables us to study regulatory networks underlying phenotypes.
This led me to study genetic regulatory relationships in health and diseases by identifying allele-specific expression and allele-specific open chromatin.

Advances in omics methods also offer opportunities to identify and study phenotype- or disease-associated genetic variants using methods like genome-wide association studies (GWAS).
GWAS generated numerous genetic associations with phenotypes and diseases.
However, it is not yet possible to interpret most of these findings due to limited biological knowledge.
Therefore, dissecting molecular functions of GWAS variants is important in the post-GWAS era.

Importantly, integrating multi-omics data can answer: How does information flow in biological systems?
The integration is biologically informative as it represents the biological signals flowing underlying phenotypes of interest or disease conditions.
Moreover, single-cell methods deepen our biological knowledge by capturing characteristics and functions per cell.

I explored genetic determinants of human molecular traits - allelic imbalance and host responses to pathogenic viruses in this thesis.

% Dutch version translated by https://www.deepl.com
Dit proefschrift beoogt de genetische basis van complexe menselijke eigenschappen te begrijpen met systeembiologie (SB).
SB maakt gebruik van kwantitatieve metingen, computationele modellen, en high-throughput screening om biologische systemen te ontcijferen.

Next-generation sequencing technologieën maken het mogelijk menselijke moleculaire eigenschappen kosteneffectief op grote schaal te profileren, waardoor het biologieonderzoek in het multi-omics tijdperk is beland.

Omics-studies zijn begonnen met genomics.
Door genomics te combineren met transcriptomics en epigenomics, kunnen we regelgevende netwerken bestuderen die aan fenotypes ten grondslag liggen.
Dit heeft mij ertoe gebracht genetische regulatierelaties in gezondheid en ziekten te bestuderen door allelspecifieke expressie en allelspecifiek open chromatine te identificeren.

Vooruitgang in omics-methoden biedt ook mogelijkheden om fenotype- of ziekte-geassocieerde genetische varianten te identificeren en te bestuderen met behulp van methoden zoals genoomwijde associatiestudies (GWAS).
GWAS hebben talrijke genetische associaties met fenotypes en ziekten opgeleverd.
Door de beperkte biologische kennis is het echter nog niet mogelijk om de meeste van deze bevindingen te interpreteren.
Daarom is het ontleden van de moleculaire functies van GWAS-varianten belangrijk in het post-GWAS tijdperk.

Belangrijk is dat de integratie van multi-omics data een antwoord kan geven op de volgende vragen: Hoe stroomt informatie in biologische systemen?
De integratie is biologisch informatief omdat het de biologische signalen weergeeft die ten grondslag liggen aan interessante fenotypes of ziektebeelden.
Bovendien verdiepen single-cell methoden onze biologische kennis door kenmerken en functies per cel vast te leggen.

In dit proefschrift heb ik genetische determinanten van menselijke moleculaire eigenschappen onderzocht - allelische onevenwichtigheid en gastheerreacties op pathogene virussen.

\end{document}
