\documentclass{book}

\begin{document}
% \marginwatermark{\thechapter}{-160}
\renewcommand{\thetable}{\textbf{Table \arabic{chapter}.\arabic{table}}}
\renewcommand{\thefigure}{\textbf{Figure \arabic{chapter}.\arabic{figure}}}

\begin{refsection} % To make a local bibliography

% \chapter*{\textit{ASdeep} uses a deep-learning neural network to attribute allelic expression to non-coding genetic variants}
% Zhenhua Zhang \textsuperscript{1,2}, Herbert T. Kruitbosch \textsuperscript{3}, Morris A. Swertz \textsuperscript{1,2}, and K. Joeri van der Velde \textsuperscript{1,2}
%
% \textsuperscript{1} Genomics Coordination Center, University of Groningen and University Medical Center Groningen, Antonius Deusinglaan 1, 9713 AV, Groningen, The Netherlands \par
% \textsuperscript{2} Department of Genetics, University of Groningen and University Medical Center Groningen, Antonius Deusinglaan 1, 9713 AV, Groningen, The Netherlands \par
% \textsuperscript{3} Center for Information Technology, University of Groningen, Nettelbosje 1, 9747 AJ Groningen, The Netherlands \par

\chapter{\textit{ASdeep} uses a deep-learning neural network to attribute allelic expression to non-coding genetic variants}
\textbf{Zhenhua Zhang}, Herbert T. Kruitbosch, Morris A. Swertz, and K. Joeri van der Velde
\vfill
\begin{flushright}
\textit{Draft is ready}
\end{flushright}

\newpage

\section*{Abstract}
\paragraph*{Summary}
  Many studies have shown that non-coding variants (NCV) affect phenotypes via regulation.
  Estimation of allelic difference is thus a novel way to identify underlying mechanisms through the potential regulating programmes.
  However, existing variant annotation tools do not incorporate allelic differences well.
  We therefore developed \textit{ASdeep}, a tool that attributes allelic differences such as allele-specific expressions to NCV that are potentially involved in the regulation.

\paragraph*{Availability and Implementation} PyPi (\url{https://pypi.org/project/asdeep}) and GitHub (\url{https://github.com/zhenhua-zhang/asdeep}).
% \paragraph*{Contact}: k.j.van.der.velde@umcg.nl (UMCG email)
% \paragraph*{Supplementary information}: Supplementary data are available at Bioinformatics online.
% The test dataset is available on Zenodo (https://zenodo.org/xxx).

\newpage
\section*{Introduction}
Diploid organisms inherit two sets of genetic information from their parents.
This introduces allelic differences that can consequently result in distinct behaviors between alleles like allele-specific expression (ASE).
In ASE, the transcripts from the maternal and paternal alleles have different abundances \cite{Pastinen2010Genome}.
In the past few decades, researchers have demonstrated important roles for ASE in many human phenotypes and diseases, including aging \cite{Balliu2019Genetic}, tumor development \cite{Liu2018A,Przytycki2020Differential} and Parkinson’s disease \cite{Langmyhr2021Allele}.
Apart from well-known mechanisms such as chromatin openness and random monoallelic expression, studies have also suggested that the differential regulation caused by non-coding variants (NCV) interferes with the expression balance of allelic transcripts \cite{Bader2015Negative,Ge2009Global,Pickrell2010Understanding,Lefebvre2012Genotype}.
Linking NCV to ASE can thus provide new insights into the regulatory mechanisms of human traits such as diseases.
However, existing tools are limited to a small number of NCV due to the computational capacity of the algorithms implemented.
Here, we introduce \textit{ASdeep}, a command-line tool based on a state-of-the-art convolutional neural network (CNN) to predict the consequence of NCV that explicitly focusses on ASE.
\textit{ASdeep} uses Markov chain–Monte Carlo (MCMC) methods to estimate ASE using allelic read counts quantified from RNA sequencing data.
The model can be used to predict ASE from a DNA sequence in the absence of RNA data to enable targeted follow-up transcriptomics.
\textit{ASdeep} can also pinpoint hot-spot genomic regions that harbor potentially causal NCV using feature-attribution methods \cite{Explainable2020Xie}, which can aid in exploring the architecture of genomic regulations.

\section*{Implementation}
The \textit{ASdeep} package was written in Python and built upon \textit{PyTorch}, a sophisticated deep-learning framework, to work as a downstream analysis tool after quantifying reads using software like ASEReadCounter from GATK \cite{Castel2015Tools}.
\textit{ASdeep} has four main functions: (i) to infer ASE, (ii) to create the train/test database, (iii) to train the CNN model and (iv) to predict/attribute the ASE effects to NCV.
To estimate the ASE (target) of a transcript, \textit{ASdeep} assigns allelic read counts to each allele based on phased genotypes at the heterozygous positions, per individual.
The imbalance is then estimated by Bayesian estimation using the MCMC approach, assuming the number of allelic read counts follows a Beta-Binomial distribution.
Next, \textit{ASdeep} transforms the DNA sequence into an image-like matrix (a two-dimensional Hilbert curve) following the same steps introduced in a previously published work \cite{Image2018Yin}.
However, we did not use one-hot encoding methods but rather directly used the index of the 4-mers in the k-mers list, where the order of the k-mers list follows the alphabetic order of the bases, i.e., A, C, G, T (see the online document of \textit{ASdeep} for more details).
To quickly and conveniently access the training, testing and unseen samples, \textit{ASdeep} stores the ASE effects (for training and testing) and the Hilbert curve of a transcript in an HDF5 database.
We then train a CNN model with pre-built architecture (e.g., \textit{AlexNet} and \textit{ResNext}) using sample information loaded from the HDF5 database.
Finally, the prediction function outputs the probability (P\textsubscript{ASE}) of ASE.
Moreover, it can also attribute ASE to genomic regions that harbor genetic variants involved in regulating the allelic expression of the transcript.

\section*{Usage}
\textit{ASdeep} has a text user-interface with four sub-commands - \textit{inferai}, \textit{makedb}, \textit{train} and \textit{predict} - that corresponds to the four key steps.
In the current implementation, we designed \textit{ASdeep} as a downstream tool of allelic read counts from \textit{ASEReadCounter}.
The first step is inference of ASE by \textit{inferai} using the allelic read counts per sample.
This generates a report with the estimated PASE, the corresponding highest density interval (HDI) and allelic read counts at the heterozygous positions for each transcript of a sample.
Users can then select a transcript of interest and collect ASE effects for each individual into a plain text report (ASE reports).
A list of canonical transcripts compiled from the ENSEMBL database \cite{Howe2020Ensembl} is available (see Data Availability).
Using the \textit{makedb} sub-command, an HDF5 database can be created that includes the reference genome sequence, genetic variants, compiled ASE reports and genomic regions that are upstream to the transcript of interest. 
The \textit{train} sub-command allows training of a CNN model with pre-built architecture using the Hilbert curve stored in the HDF5 database.
Finally, we implemented two functions in the predict sub-command: 1) to predict the probability of ASE and 2) to attribute the ASE effects to a genomic sequence (variants).
A detailed example is available in the Supplementary Material, and the corresponding test dataset is hosted on Zenodo (see Data Availability).
We created the test dataset via an in-house pipeline using \textit{fastp} \cite{Chen2018fastp} for quality control, \textit{STAR} \cite{Dobin2012STAR} for alignment, \textit{WASP} \cite{Hoek2021WASP} to remove mapping bias and \textit{ASEReadCounter} \cite{Castel2015Tools} to count allelic reads.

\section*{Conclusion}
We have developed a deep-learning tool, \textit{ASdeep}, that can predict ASE from NCV and attribute these variants to the divergence of alleles.
As a starting point, \textit{ASdeep} first estimates the ASE effects of a transcript from allelic read counts using Bayesian estimation and MCMC.
The tool then trains a CNN model with pre-built architecture on the genomic context of the transcript.
Finally, \textit{ASdeep} can predict ASE effects and attribute them to hot-spot genomic regions that potentially harbor variants involved in regulation of gene expression.
\textit{ASdeep} is also applicable to other allele-specific data, such as allele-specific methylation and allele-specific open chromatin, where allelic read counts are available.

\section*{Acknowledgments}
We thank the UMCG Genomics Coordination Center, the UMCG Research IT programme and the UG Center for Information Technology and their sponsors Biobanking and Biomolecular Research Infrastructure Netherlands (BBMRI-NL) \& TarGet for storage and compute infrastructure.
We thank the Biobank-Based Integrative Omics Studies (BIOS) Consortium, funded by BBMRI-NL, a research infrastructure financed by the Netherlands Organization for Scientific Research (NWO) under award number 184.021.007.
The BIOS Consortium members are listed in Supplementary Data S1.
We thank the Genotype-Tissue Expression (GTEx) Project, supported by the Common Fund of the Office of the Director of the National Institutes of Health (commonfund.nih.gov/GTEx).

\section*{Funding}
Z.Z. received a joint scholarship from the University Groningen and China Scholarship Council (under grant number 201706350277).
Z.Z., K.J.v.d.V and M.A.S. received funding from the Netherlands Organisation for Scientific Research (NWO) under VIDI grant number 917.164.455.
In addition, we acknowledge support from the European Union's Horizon 2020 research and innovation programme under grant agreement No. 779257 (Solve-RD) and 825575 (European Joint Programme on Rare Disease).

\section*{Data availability}
The source code of \textit{ASDeep} is publicly available on GitHub (\url{https://github.com/zhenhua-zhang/ASdeep}), and its release versions are available at PyPi (\url{https://pypi.org/project/ASdeep}).
% Along with the package, a testing dataset derived from the GEUVADIS project \cite{Lappalainen2013Transcriptome} is available at Zenodo (\url{https://zenodo.org/xxx})


% Reference list
\section*{References}
\printbibliography[heading=none]

\clearpage

\end{refsection}
\end{document}

% vim: set tw=500:
